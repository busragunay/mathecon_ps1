\documentclass[10pt,letter]{article}
	% basic article document class
	% use percent signs to make comments to yourself -- they will not show up.

\usepackage{amsmath}
\usepackage{amssymb}
\usepackage{multicol}
	% packages that allow mathematical formatting

\usepackage{graphicx}
	% package that allows you to include graphics

\usepackage{setspace}
	% package that allows you to change spacing

\onehalfspacing
	% text become 1.5 spaced

\usepackage{fullpage}
	% package that specifies normal margins
	

\begin{document}
	% line of code telling latex that your document is beginning

\newcommand{\horrule}[1]{\rule{\linewidth}{#1}} % Create horizontal rule command with 1 argument of height

\title{  \horrule{0.5pt} \\[0.4cm] Mathematical Economics}

\author{ECON 344}

\date{ASSIGNMENT 1}
	% Note: when you omit this command, the current dateis automatically included


	% tells latex to follow your header (e.g., title, author) commands.
\maketitle
\horrule{2pt} \\[0.5cm] % Thick bottom horizontal rule
\section*{Problem 1}

Consider a demand function $Q_d=f(P)$ where $f'(P)<0$. Show that $ \displaystyle \frac{dR}{dP}=Q(1- \left| \epsilon_P \right|)$ where $R$ denotes total revenue and $\epsilon_P$ is the price elasticity of demand. \vspace{70mm}


\section*{Problem 2}

A firm with the following demand function $Q_D=f(P)$, where $f'(P)<0$ faces a linear average revenue function (remember that $AR=\displaystyle \frac{TR}{Q}$). Show that:

\subparagraph{i)} Marginal revenue is a linear function of $Q$.  \vspace{30mm}

\subparagraph{ii)} The absolute value of the slope of the MR line is twice that of the AR line. \vspace{30mm}

\subparagraph{iii)} The $Q$ intercept of the AR line is twice the $Q$ intercept of the MR line. \vspace{30mm}




\section*{Problem 3}

Check whether the following functions are concave or convex:

\subparagraph{i)} \(y=f(x)=x^2+3x\), where \(x>0\) \vspace{20mm}

\subparagraph{ii)} \(y=f(x)=x^a\), where \(x,a>0\) \vspace{20mm}

\subparagraph{iii)} \(y=f(x)=e^{2x+1}\), where \(x>0\) \vspace{20mm}

\subparagraph{iv)} \(y=f(x)=bx^a\), where \(x,b>0\) and \(a<0\)\vspace{20mm}



\section*{Problem 4}

Calculate the following marginal values.

\subparagraph{i)} Marginal product of labor (MPL) for the production function; \(Q=F(L)=L^{0.5}\) \vspace{20mm}

\subparagraph{ii)} Marginal product of capital (MPK) for the production function  \(Q=F(L,K)=L^{0.4}K^{0.6}\) \vspace{20mm}

\subparagraph{iii)} Marginal utility of income, where \(U(I)=5I^{0.8}\) \vspace{20mm}

\subparagraph{iv)} Marginal utility of apples, where \(U(A,B)=2lnA+B^{0.3}\) \vspace{20mm}

\subparagraph{iv)} Marginal cost of production, where the total cost function is  \[TC(q)=q^2-3q+400\] \vspace{20mm}

\newpage

\section*{Problem 5}

Calculate the following elasticities.

\subparagraph{i)} Price elasticity of demand at $p=10$, for the demand function: \(Q=100-2P\) \vspace{20mm}

\subparagraph{ii)} Income elasticity of demand at $I=50$, for the demand function: \[Q=100+ I^{0.4}-4P\] \vspace{20mm}

\subparagraph{iii)} Cross-Price elasticity of demand at $p_x=5$ and $p_y=2$, for the demand function: \[Q_x=100-2p_x+4p_y+P_xP_y\] \vspace{20mm}

\subparagraph{iv)} Price elasticity of supply at $p=10$, for the supply function: \(Q=aP^b\) \vspace{20mm}

\subparagraph{iv)} Price elasticity of supply at $p=10$, for the supply function of the log-form: \[\ln{Q}=a+b\ln{P}\] \vspace{20mm}

\end{document}
	% line of code telling latex that your document is ending. If you leave this out, you'll get an error
